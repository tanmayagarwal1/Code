\documentclass{article}
\newcommand{\nd}{\noindent}
\usepackage{amsmath} 
\parskip 3mm 
\begin{document}
\title{Bio inspired algorithm and project details}
\author{Tanmay Agarwal} 
\date{16 MAR, 2021}
\maketitle 
\tableofcontents
\section{Introduction and Inspiration}
Whenever mankind has shown pride and pretension on its work, nature has always taught it a lesson. On 10th april, 1912, the Rms(Royal Mail Ship) Titanic set sail into the vast atlantic ocean. Titanic was termed as the "unsinkable" ship. The engineers and architects who designed the vessel were so swollen up on their pride that they skimped many safety measures and deployed less life boats on the titanic as it was, the unsinkable. On the night of 14th april, titanic hit an iceberg in the atlantic and sunk. What was the cost of making an unsinkable skip you ask ? 1500 dead including men, women and children. The pride of making an unsinkable ship resulted in one of the most deadliest incidents in the history of civilian transportation and the deadliest peacetime sinking of a superliner or cruise ship to date. 

\nd This is the major inspiration for our project. After this incident a majority of the work was focused on risk and disaster mitigation for civilian travel methods. We want to develop and deploy a model, that can help the disaster management authorities to mitigate the risk and increase the probability of surviving of the passengers on board a civilian vessel. Even if our model is able to increase the survival rate incase of a disaster, it will be an honour to have worked on it. Our model predicts the probability of survival of each passenger on board using certain attributes and metrics from a given dataset and this probability can be used by risk and disaster management authorities to facilitate the appropriate amounts of safety precautions on board. 

\section{Bio inspired alogrithm} 
Nature has been a source of inspiration for many great endeavours and what better to teach us survival better than the nature itself. Survival instinct can be visualised among a group of animals by observing their behaviour and the way animals in the environment communicate with each other in groups and flocks. Bio-inspired computing optimization algorithms is an emerging approach which is based on the principles and inspiration of the biological evolution of nature to develop new and robust competing techniques. In the last years, the bio-inspired optimization algorithms are recognized in machine learning to address the optimal solutions of complex problems in science and engineering. To tackle the problems of the traditional optimization algorithms, the recent trends tend to apply bio-inspired optimization algorithms which represent a promising approach for solving complex optimization problems.

\nd One of the other main reasons to deploy bio inspired algorithms is that they are compuatationally less expensive. When compared to traditional algorithms like the Stochastic Gradient Descent algorithm, which uses first and second order partial differential equations for search optimisation, bio inspired algorithms use linear algebra which is not only less expensive on the syste, hardware but also faster to compute even on older electronic chipsets

\section{Particle swarm optimisation Algorithm} 
Particle Swarm Algorithm is a metaheuristic (Problem independent), bio inspired search optimsiation algorithm. In the early of 1990s, several studies regarding the social behavior of animal groups were developed. These studies showed that some animals belonging to a certain group, that is, birds and fishes, are able to share information among their group , and such capability confers these animals a great survival advantage. Inspired by these works, Kennedy and Eberhart proposed in 1995 the PSO algorithm. particle swarm optimization (PSO) is a computational method that optimizes a problem by iteratively trying to improve a candidate solution with regard to a given measure of quality.

\nd To better get a visualisation of this algorithm consider the following scenario. A swarm of birds flying over a place must find a point to land and, in this case, the definition of which point the whole swarm should land is a complex problem, since it depends on several issues, that is, maximizing the availability of food and minimizing the risk of existence of predators. In this context, one can understand the movement of the birds as a choreography; the birds synchronically move for a period until the best place to land is defined and all the flock lands at once.

\nd The main parts of an PSO algorithm are the Position Vector, Momentum Vector, Fitness Function, Particles best position, Global Best position. The position vector is the goal of the optimisation problem, The fitness function is the objective function which determines the position, the velocity vector represents the speed at which the swarm moves. The updation occurs as follows : 
\begin{equation}
X^{t+1}_{ij}= X^{t}_{ij}+ V^{t+1}_{ij} 
\end{equation} 
\begin{equation}
V^{t+1}_{ij}= wV^{t}_{ij}+c_1 r^{t}_{1} (pbest_{ij}-X^{t}{ij})+c_2 r^{t}_{2}(gbest_j - X^{t}_{ij})
\end{equation} 

\section{Input and Output} 
The input to our model is the dataset and output is the probability of survival of a passenger on the basis of attributes and metrics provided by the dataset, 

\end{document}
