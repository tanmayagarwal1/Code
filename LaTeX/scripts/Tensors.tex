\documentclass{article}
\usepackage{amsmath}
\begin{document}
\normalsize
\title{TENSOR WORKSHEET}
\author{TANMAY AGARWAL}
\date{\today}
\maketitle
\begin{abstract}
In this paper we will be convering most of the tensors and their various transformations. We will also take a look on the dirac notation formalizations
\end{abstract}
\tableofcontents
\section{Initial Tensor And Their Transformations}
Firstly we will be looking at the tensors which have only one index. Broadly as we all know there are two types of tensors 
\begin{equation*}
\text{1.Contra-variant tensors}
\end{equation*}
\begin{equation*}
\text{2.Co-variant tensors}
\end{equation*}
In contra-variant tensors the components transfrom in a "contra" manner. If the components increase, their respective products will decrese(If still any confusion please refer the main notes).
The various Transformations of contra-variant vectors are given as 
\begin{equation}
X^{'i}=\frac{\partial X^{'i}}{\partial X^j}\cdot X^j
\end{equation}
In co-variant tensors the components trnsform in a "co" manner. If the components increase or decrease, even the products increases or decreses. Their tansformation is 
\begin{equation}
X_i^{'}=\frac{\partial X^j}{\partial X_i^{'}}\cdot X^j
\end{equation}
These transformations can also be representes as follows 
\begin{equation}
X^{,i}=\wedge^{ij}\cdot X^j
\end{equation}
and 
\begin{equation}
X_i^{'}=\wedge^{ji}\cdot X^j
\end{equation}
here 
\begin{equation}
\wedge^{\alpha \beta}=\frac{\partial X^{'\alpha}}{\partial X^\beta}
\end{equation}
\section{Transformations For Contra Tensors}
For contra-variant tesnors a list of various transformations are
\begin{equation}
x^{'i}=\frac{\partial x^{'i}}{\partial x^j}\cdot x^j \,\,\text{(or)}\,\, x^{'i}=\wedge^{ij}\cdot x^j 
\end{equation}

\begin{equation}
A^\alpha = \frac{\partial x^{' \alpha}}{\partial x^\beta}\cdot A^\beta 
\end{equation}

\begin{equation}
T^{\alpha \beta}=\frac{\partial x^{' \alpha}}{\partial x^j}\frac{\partial x^{'\beta}}{\partial x^j}\cdot T^{ij} 
\end{equation}
or 
\begin{equation*}
T^{\alpha \beta}=\wedge^{\alpha i}\wedge^{\beta j}\cdot T^{ij}
\end{equation*}
Proceeding
\begin{equation}
T^{\alpha \beta}=\frac{\partial y^{\alpha}}{\partial x^i}\frac{\partial y^{\beta}}{\partial x^j}\cdot T^{ij}
\end{equation}
here
\begin{equation*}
y^{\alpha}=x^{'}
\end{equation*}
Moving on, a mixed tensor is defined as
\begin{equation}
T_{\beta}^{\alpha}=\frac{\partial x^{ ' \alpha}}{\partial x^i} \frac{\partial x^j}{\partial x^{' \beta}}\cdot T_j^{i}
\end{equation}
\section{Transformations For Co-Tensors}
For Co-variant tensors the list of various transformations are 
\begin{equation}
x_i^{'}=\frac{\partial x^j}{\partial x^{'i}}\cdot x_j
\end{equation}

\begin{equation}
A_i=\frac{\partial x^j}{\partial x^{'i}}\cdot A_j
\end{equation}

\begin{equation}
T_{\alpha \beta}=\frac{\partial x^i}{\partial x^{'\alpha}} \frac{\partial x^j}{\partial x^{'\beta}}\cdot T_{ij}
\end{equation}
\begin{equation}
\Gamma_{\mu \nu}^{\alpha}= \frac{\partial x^\lambda}{\partial x^\alpha} \cdot \frac{\partial^2 x^\alpha}{\partial x^\mu \partial x^\nu}
\end{equation}
\begin{equation}
\sum_{i}{A_i}
\end{equation}
\end{document}