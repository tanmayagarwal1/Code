\documentclass{article}
\usepackage{amsmath}
\usepackage{braket}
\begin{document}
\title{Angular Momentum Complete}
\author{TANMAY AGARWAL}
\date{\today}
\maketitle
\tableofcontents
\section{Formalization}
In this frame we will be formalizing the the angular momentum opertors in 3D.
First, we know that in general 
\begin{equation*}
L=R\times{P}
\end{equation*}
From this we formalize the equations for our angular momentum to be 
\begin{equation}
L_x=Yp_z \times Zp_y
L_y=Zp_x\times Xp_z
L_z=Xp_y\times  Yp_x
\end{equation}
also we define the total angular momentum to be 
\begin{equation}
\braket{L}^2=\braket{x}^2+\braket{y}^2+\braket{z}^2
\end{equation}
The commution relations are as follows 
\begin{equation}
[L^2,L_z]=0 
[L_x,L_y]=i\hbar
\end{equation}
From the above, we conclude that there exists common eigen functions for $\braket{L}^2$ and $\braket{L_z}$
Let $Y_{lm}$ be the common eigen function of them both. The commutation relations of $Y_{lm}$ is 
\begin{equation}
[Y_{lm},\braket{L}^2]=0 
\end{equation}
Also we define two non Hermitian ladder operators due to the commutation relations as follows
\begin{equation}
L_{\pm}=L_x \pm L_y 
\end{equation}
Now the commutation relation of $Y_{lm}$ with our ladder operators are
\begin{equation}
[Y_{lm},L_{\pm}]=\hbar m
\end{equation}
Before moving on lets layout the eigenValues of $Y_{lm}$ when acted upon by the opertors 
\begin{equation}
L^2Y_{lm}=\hbar^2l(l+1)
\end{equation}
\begin{equation}
L_zY_{lm}=\hbar mY_{lm}
\end{equation}
from this we define the followling relations 
\begin{equation}
\text{Let $\psi = L_{\pm}Y_{lm}$}
\end{equation}
Hence from here we define the following
\begin{equation}
\braket{L}^2 \psi = L^2  L_{\pm} Y_{lm}
\end{equation}

\begin{equation}
 [[L^2,L_{\pm}]+L{\pm}]Y_{lm}
\end{equation}

\begin{equation}
 [0]+L_{\pm}\hbar^2 l(l+1)Y_lm
\end{equation}
From this we see that it$ L^2$ has not effect on the wavefunction if its acted on by $Y_{\pm}$. 
Now the next thing is 
\begin{equation}
L_z\psi=L_zL_{\pm}Y_{lm}
\end{equation}

\begin{equation}
[[L_z,L_{\pm}]+L_{\pm}L_z]\,Y_{lm}
\end{equation}

\begin{equation}
[\hbar L_{\pm}+L_{\pm}\hbar m]\,Y_{lm}
\end{equation}

\begin{equation}
\hbar [m+1]L_{\pm}Y_{lm}
\end{equation}

\begin{equation}
 \hbar [m+1] \psi(x)
\end{equation}
\section{The Ladder Structure}
Now we would like to find the possible values of  $m$. To do this we use the expectation values of the opertors.
\begin{equation}
\bra{\psi}L^2\ket{\psi}=\bra{\psi}L_{x}^2\ket{\psi}+\bra{\psi}L_{y}^2\ket{\psi}+ \bra{\psi}L_{z}^2\ket{\psi}
\end{equation}
This yeilds
\begin{equation}
\hbar^2l(l+1) > \hbar^2 m^2
\end{equation}
Solving this quadratic equation yields the following 
\begin{equation}
m_+= l_+ 
\end{equation}
\begin{equation}
m_-= l_-
\end{equation}
This means the maximum value of $m$ is $l_+$ and minimum is $l_-$
We hence find out that in the ladder of Momentum eigen values, the number of states $N$ is 
\begin{equation}
N=2l+1
\end{equation}
which means that $l$ can only take integer or half int values
\begin{equation}
l=1,1/2,2/3,2,....
\end{equation}
NOTE : say$ L = 1$ and $L_z= 1$ . This still does not mean that $L_x$ and $L_y$ will be zero. Lets say in general that $l=m$. Finding the expected values we get 
\begin{equation}
\hbar^2 l(l+1)= \bra{\psi}L_{x}^2\ket{\psi}+\bra{\psi}L_{y}^2\ket{\psi}+\hbar^2 m^2
\end{equation}
Now as $l=m$
\begin{equation}
\bra{\psi}L_{x}^2\ket{\psi}+\bra{\psi}L_{y}^2\ket{\psi}=\hbar^2 l^2+\hbar^2 l - \hbar^2 m^2
\end{equation}
\begin{equation}
\bra{\psi}L_{x}^2\ket{\psi}+\bra{\psi}L_{y}^2\ket{\psi}=\hbar^2 l
\end{equation}
If let: The expectation values of $L_x$ and $L_y$ be equal. Then 
\begin{equation}
\bra{\psi}L_{x}^2\ket{\psi}=\frac{\hbar^2 l}{2} \text{which is} >0
\end{equation}
Hence even if $ L_z=L^2$ there is still non zero expectation values in both $L_x$ and $L_y$
\section{The 3 Dimensions}
Now let us move on to discussing the Spherical Harmonics part of angular momentum which is exciting. 
In here $Y_{lm}$ is a funtion of $\phi$ and $\theta$ . $\phi$ is the angle made at the $x-y$plane and $\theta$ with the $z$ axis. 
Now our operstors in 3D are
\begin{equation}
L_z=\frac{\hbar }{i}\partial_\phi \text{or} L_\phi=-i\hbar\partial_\phi
\end{equation}
\begin{equation}
L_{\pm}={\pm}\hbar e^{{\pm}i\phi}(\pm \partial_\theta + \cot{\theta}\partial_{\phi})
\end{equation}
Now  if we use the fact that 
\begin{equation}
L_zY_{lm}=\hbar mY_{lm}
\end{equation}
Substituting the value of $L_z$ we get 
\begin{equation}
Y_{lm}(\theta,\phi)=e^{im\phi}P_{l}(\theta)
\end{equation}
Where 
\begin{equation}
P_l(\theta)=\text{some unknow dependence on $\theta$}
\end{equation}
Now to proceed from here we establish the fact that at $\phi=2\pi$ and $\phi=0$ hence 
\begin{equation}
Y_{lm}{\theta,0 }=Y_{lm}(\theta,2\pi)
\end{equation}
This is because after one complete rotation it must come back to the same point. 
Now 
\begin{equation}
Y_{lm}{\theta,0)=e^{im0}P_l(\theta) = P_l(\theta}
\end{equation}
Similarly 
\begin{equation}
Y_{lm}(\theta,2\pi)=
\begin{cases}
P_l(\theta) \text{if $m=int$}
\\
-P_l(\theta) \text{if $m=$half ints}
\end{cases}
\end{equation}
Now the second case should notbe possible. Hence we conculde that 
\begin{equation}
\text{$m$ cannot have half int values. Hence $l$ is strictly an integer}
\end{equation}
\section{The WaveFunction In 3D}
Having established this fact we know that $L_+$ acting on $m_+$ will be equal to $0$. Hence by substitution 
\begin{equation}
{\pm}e^{im\phi}(\partial_\theta \pm \cot{\theta}\partial_{\phi})(m_+)=0
\end{equation}
We find the solutions of this to be associated with legendre polynomials. Hence the properlt NORMALIZED wvefunction is 
\begin{equation}
Y_{lm}(\theta,\phi)=(-1)^m \sqrt{\frac{2l+1}{4\pi} \frac{(|l|-|m|)!}{(|l|+|m|)!}}e^{im\phi}P_l^m(\cos{\theta}) 
\end{equation}
where $P_l^m(x)=$ associated legendre polynomial and in that $P_l(x)=$ the legendre polynomial solutions and $x=\cos{\theta}$
\begin{equation}
P_l^m(x)=(1-x^2)^{\frac{|m|}{2}}\,\frac{\partial^m}{\partial_x^m}p_l(x) 
\end{equation}
\begin{equation}
P_l(x)=\frac{1}{2^x x!}\, \frac{\partial^x}{\partial_x^x}(x^2-1)^x
\end{equation}
And the normalization conditon used is 
\begin{equation}
\int_{0}^{\pi}\int_{0}^{2\pi}\int_{-\infty}^{\infty}{|Y_{lm}(\theta,\phi)|^2}= 1
\end{equation}
\section{The hadronic and leptonic tensor}
This is a formal introduction to the leptonic and hadronic tensor. 
\begin{equation}
d\sigma = \frac{1}{4ME}\,\, \frac{d^3 k}{2\pi^{3} 2E}\,\, \frac{d^{3} p }{2\pi^{3}2p^{'}}\,\,\bigg\{\frac{e^{4}}{q^{4}L_{e}^{\mu \nu}} L_{\mu \nu}^{\text{muon}}\bigg\}\, (2\pi)^{4} \sigma^{4} (p+q-p^{'})
\end{equation}
This implies that the hadronic tensor, which is of a covariant nature can be written as : 
\begin{equation*}
W_{\mu \nu}= \frac{1}{4\pi M}\,\, \Big( \frac{1}{2} \sum_{s} \sum_{s^{'}} \Big) \int{\frac{d^{3}p{'}}{(2\pi)^{3} 2p^{'}} \bra{p,s}J_{\mu}^{+}\ket{p^{'},s^{'}} \\
 \textbf{X} \bra{p^{'},s^{'}}J_{\mu}^{+}\ket{p,s} (2\pi)^{4}(p+q-p^{'})}
\end{equation*}





\end{document}